In recent years, there has been a growth of interest in data visualization technologies for human-assisted data analysis using systems such as [1,10,11]. While computers can provide high-speed and high-volume data processing, humans have domain knowledge and the ability to process data in parallel, notably by using our visual systems. Most importantly, humans provide the definition of what is valuable in an analysis. Accordingly, human/computer analytic systems are essential to extracting knowledge of value to humans and their societies from the large amounts of data being generated today.

\subsection{Interactivity}
Visualization systems are most effective when they are interactive, thereby allowing a user to explore data and connect it to their domain knowledge and sense of what is important without breaking cognitive flow. In recent years, a number of such systems have been developed, both by the academic community and by the commercial sector. Exploration of data consists not only in creating visual displays, but also in creating and modifying domain-specific computations. Consequently, most data visualization systems include facilities for defining such calculations as part of the data model being analyzed. The most effective systems allow users to define these calculations as part of the analytic interaction, which permits the user to stay in the flow of analysis [9].

During the analytic process, a user may discover that parts of the data are not yet suitable for analysis. Solutions to this problem are often provided by data preparation tools external to the visual analysis environment, which requires the user to break their analytic flow, launch another tool and reprocess their data before returning to their analysis. If the user does not own this process (e.g. it is the responsibility of another department), then there can be significant delays (including ``never.") More subtly, the result of updated external processing may not be compatible with the user's existing work, which can lead to more time lost reconciling the new data model with the existing analysis.

From the user's perspective, the boundary between preparation and analysis is not nearly so clean cut. Bad data is often discovered using visual analysis techniques (e.g. histograms or scatter plots) and it is most natural for the user to ``clean what she sees" instead of switching to a second tool. This leads to an ``adaptive" process whereby users will prefer leveraging existing tools in the analytics environment (no matter how well suited to the task) over switching to another application. Thus a well-designed interactive visual analysis environment will provide tools that enable users to perform such cleaning tasks as interactively as possible.
