\subsection{Context-Free Grammar}
ICU date-time formats are well defined both structurally and semantically, and can be defined by a context-free grammar (CFG). A CFG is commonly defined as a set of productions or rules of the form A ? ? where A is a variable, ? is a sequence of variables and terminal symbols (the tokens that make up the alphabet of the language) plus null (? ), and the production symbol (?) indicates that the variable A can be expanded into ? . A CFG can be formally specified with four components: V, T, P, and S, where V is the set of variables, T the set of terminal symbols, P the set of productions, and S the set of available start symbols (a non-empty subset of V) [cite]. 

Pattern-recognition problems such as parsing date and time formats initiate from observations generated by some structured stochastic process. In other words, even if the initial higher-level production rule of the grammar is known (i.e. date, time or date-time), there could be several directions that the parser resolve to. For example, in a date string 5/6/2015, the pattern could either be M/d/yyyy or d/M/yyyy. 

Probabilistic context-free grammars (PCFGs) have provided a useful method for modeling such uncertainty [cite]. Once we have created a PCFG model of a process, we can apply existing PCFG parsing algorithms to identify a variety of date-time formats. However, the parser�s success is often limited in the types of the dominant patterns that it can identify. In addition, the standard parsing techniques generally require specification of a complete observation sequence. In many contexts, we may have only a partial sequence available (e.g. an incomplete entry). Finally, we may be interested in computing the probabilities of date-time patterns that the grammar may not explicitly define. To extend the forms of evidence, inferences, and pattern distributions supported, we need a flexible and expressive representation for the distribution of structures generated by the grammar. We adopt Bayesian networks for this purpose, and define an algorithm to generate a probabilistic distribution of possible parse trees corresponding to a set of date-time patterns as opposed to individual ones. 


%�	Motivate the use of context-free grammar, particularly Backus-Normal-Form
%�	Explain why we use a probabilistic version of the grammar
%�	Explain how the production rules are created based on the ICU format
%�	Extensions to the parser to account for columnar context for maximizing the probabilistic occurrence of the dominant pattern
%�	Other semantic extensions � looking at the file name, adding rules for leap year, using external dictionaries for non-English terms
