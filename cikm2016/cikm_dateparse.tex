% THIS IS SIGPROC-SP.TEX - VERSION 3.1
% WORKS WITH V3.2SP OF ACM_PROC_ARTICLE-SP.CLS
% APRIL 2009
%
% It is an example file showing how to use the 'acm_proc_article-sp.cls' V3.2SP
% LaTeX2e document class file for Conference Proceedings submissions.
% ----------------------------------------------------------------------------------------------------------------
% This .tex file (and associated .cls V3.2SP) *DOES NOT* produce:
%       1) The Permission Statement
%       2) The Conference (location) Info information
%       3) The Copyright Line with ACM data
%       4) Page numbering
% ---------------------------------------------------------------------------------------------------------------
% It is an example which *does* use the .bib file (from which the .bbl file
% is produced).
% REMEMBER HOWEVER: After having produced the .bbl file,
% and prior to final submission,
% you need to 'insert'  your .bbl file into your source .tex file so as to provide
% ONE 'self-contained' source file.
%
% Questions regarding SIGS should be sent to
% Adrienne Griscti ---> griscti@acm.org
%
% Questions/suggestions regarding the guidelines, .tex and .cls files, etc. to
% Gerald Murray ---> murray@hq.acm.org
%
% For tracking purposes - this is V3.1SP - APRIL 2009

%\documentclass{acm_proc_article-sp}
%\documentclass[conference]{IEEEtran}
\documentclass{sig-alternate-05-2015}
%
\usepackage{url}
\usepackage{syntax}
\usepackage{amsmath}% http://ctan.org/pkg/amsmath
\usepackage{mathtools}
\usepackage{color}
\usepackage{caption} 
\usepackage{tikz}
\usepackage{enumitem}
\captionsetup[table]{skip=10pt}

\newenvironment{tight_itemize}{\begin{itemize} \itemsep
-2pt}{\end{itemize}}
%
\def\sharedaffiliation{%
\end{tabular}
\begin{tabular}{c}}
%
\def\eg{\textit{e.g.\@} }
\def\ie{\textit{i.e.\@} }
\def\cf{\textit{cf.\@} }
\def\etal{\textit{et al.\@} }
\def\naive{na\"ive}
\def\Naive{Na\"ive}
\def\dateparse{\mbox{\texttt{DATEPARSE}}}
\hyphenation{time-stamp}



% to add comments
\newcommand{\vidya}[1]{\textcolor{red}{(vidya: #1)}}
\newcommand{\richard}[1]{\textcolor{green}{(richard: #1)}}
\newcommand{\dan}[1]{\textcolor{blue}{(dan: #1)}}
\newcommand\functionname[1]{\mbox{\texttt{#1}}}

% formatting math symbols
\newcommand{\Mod}[1]{\ (\text{mod}\ #1)}
\DeclarePairedDelimiter\ceil{\lceil}{\rceil}
\DeclarePairedDelimiter\floor{\lfloor}{\rfloor}

% Alter some LaTeX defaults for better treatment of figures:
% See p.105 of "TeX Unbound" for suggested values.
% See pp. 199-200 of Lamport's "LaTeX" book for details.
% General parameters, for ALL pages:
\renewcommand{\topfraction}{0.9}    % max fraction of floats at top
\renewcommand{\bottomfraction}{0.8} % max fraction of floats at bottom
% Parameters for TEXT pages (not float pages):
\setcounter{topnumber}{2}
\setcounter{bottomnumber}{2}
\setcounter{totalnumber}{2} % 2 may work better
\setcounter{dbltopnumber}{1} % for 2-column pages
\renewcommand{\dbltopfraction}{0.9} % fit big float above 2-col. text
\renewcommand{\textfraction}{0.05}  % allow minimal text w. figs
% Parameters for FLOAT pages (not text pages):
\renewcommand{\floatpagefraction}{0.8}  % require fuller float pages
% N.B.: floatpagefraction MUST be less than topfraction !!
\renewcommand{\dblfloatpagefraction}{0.8}   % require fuller float pages


%
\begin{document}


\title{Does Anybody Really Know What Time It Is?\\
Automating the Extraction of Date Scalars}
%
% You need the command \numberofauthors to handle the 'placement
% and alignment' of the authors beneath the title.
%
% For aesthetic reasons, we recommend 'three authors at a time'
% i.e. three 'name/affiliation blocks' be placed beneath the title.
%
% NOTE: You are NOT restricted in how many 'rows' of
% ``name/affiliations'' may appear. We just ask that you restrict
% the number of 'columns' to three.
%
% Because of the available 'opening page real-estate'
% we ask you to refrain from putting more than six authors
% (two rows with three columns) beneath the article title.
% More than six makes the first-page appear very cluttered indeed.
%
% Use the \alignauthor commands to handle the names
% and affiliations for an 'aesthetic maximum' of six authors.
% Add names, affiliations, addresses for
% the seventh etc. author(s) as the argument for the
% \additionalauthors command.
% These 'additional authors' will be output/set for you
% without further effort on your part as the last section in
% the body of your article BEFORE References or any Appendices.

\numberofauthors{3} %  in this sample file, there are a *total*
% of EIGHT authors. SIX appear on the 'first-page' (for formatting
% reasons) and the remaining two appear in the \additionalauthors section.
%
\author{
% You can go ahead and credit any number of authors here,
% e.g. one 'row of three' or two rows (consisting of one row of three
% and a second row of one, two or three).
%
% The command \alignauthor (no curly braces needed) should
% precede each author name, affiliation/snail-mail address and
% e-mail address. Additionally, tag each line of
% affiliation/address with \affaddr, and tag the
% e-mail address with \email.
%
% 1st. author
\alignauthor
The Doctor \\
%       \affaddr{Tableau Software}\\
%       \affaddr{837 North 34th Street}\\
%       \affaddr{Seattle, WA 98103}\\
%      \email{hawkfish@tableau.com}
% 2nd. author
\alignauthor
Martha Jones\\
%       \affaddr{Tableau Software}\\
 %      \affaddr{837 North 34th Street}\\
%       \affaddr{Seattle, WA 98103}\\
 %      \email{vsetlur@tableau.com}
% 3rd. author
\alignauthor
Captain Jack\\
%       \affaddr{Tableau Software}\\
%       \affaddr{837 North 34th Street}\\
%       \affaddr{Seattle, WA 98103}\\
%       \email{dcory@tableau.com}
%       \email{\{hawkfish,vsetlur,dcory\}@tableau.com}
\sharedaffiliation
       \affaddr{c/o Time Lords}\\
       \affaddr{The Citadel}\\
       \affaddr{Gallifrey}\\
       \email{\{doctor,martha,jack\}@tardis.org}
}
% Just remember to make sure that the TOTAL number of authors
% is the number that will appear on the first page PLUS the
% number that will appear in the \additionalauthors section.


\maketitle
\begin{abstract}
Interactive visual analytics is an effective approach for the timely analysis of big data. Users who are already engaged in interactive data analysis prefer to stay in the visualization environment for unavoidable data cleaning and preparation tasks in order to preserve their analytic flow. This has led visualization environments to include simple data preparation functions such as scalar parsing, pattern matching and categorical binning. One common scalar parsing tasks is extracting date and time data from string representations. Several relational database systems include date parsing ``mini-languages'' to cover the wide range of possible formats, but analysis of user data from one visualization system shows that the parsing language syntax is difficult for users to master without disturbing their analytic flow.

In this paper, we present two algorithms for automatically deriving date formats from a column of data with minimal user disruption, one based on minimal entropy representations and another based on natural language processing techniques. Both have accuracies of over 95\% on a large corpus of date columns extracted from an online data repository, and one is also fast enough to produce results within the user's perceptual threshold. Moreover, we were able to avoid prohibitively expensive manual verification by using the algorithms to cross-check each other at scale.
\end{abstract}

%Looks like they don't use them.
%\vidya{Need to figure out the IEEE equivalent for categories and terms}
%D.3.3 [Data Processing]: Data Cleaning, Natural Language Processing.
% A category with the (minimum) three required fields
%\category{D.3.3}{Data Processing}{Data Cleaning, Natural Language Processing}
%A category including the fourth, optional field follows...
%\category{D.2.8}{Software Engineering}{Metrics}[complexity measures, performance measures]

%\terms{Algorithms, Performance}

\section{Introduction} \label{sec:intro}
In recent years, there has been a growth of interest in data visualization technologies for human-assisted data analysis using systems such as [1,10,11]. While computers can provide high-speed and high-volume data processing, humans have domain knowledge and the ability to process data in parallel, notably by using our visual systems. Most importantly, humans provide the definition of what is valuable in an analysis. Accordingly, human/computer analytic systems are essential to extracting knowledge of value to humans and their societies from the large amounts of data being generated today.

\subsection{Interactivity}
Visualization systems are most effective when they are interactive, thereby allowing a user to explore data and connect it to their domain knowledge and sense of what is important without breaking cognitive flow. In recent years, a number of such systems have been developed, both by the academic community and by the commercial sector. Exploration of data consists not only in creating visual displays, but also in creating and modifying domain-specific computations. Consequently, most data visualization systems include facilities for defining such calculations as part of the data model being analyzed. The most effective systems allow users to define these calculations as part of the analytic interaction, which permits the user to stay in the flow of analysis [9].

During the analytic process, a user may discover that parts of the data are not yet suitable for analysis. Solutions to this problem are often provided by data preparation tools external to the visual analysis environment, which requires the user to break their analytic flow, launch another tool and reprocess their data before returning to their analysis. If the user does not own this process (e.g. it is the responsibility of another department), then there can be significant delays (including ``never.") More subtly, the result of updated external processing may not be compatible with the user's existing work, which can lead to more time lost reconciling the new data model with the existing analysis.

From the user's perspective, the boundary between preparation and analysis is not nearly so clean cut. Bad data is often discovered using visual analysis techniques (e.g. histograms or scatter plots) and it is most natural for the user to ``clean what she sees" instead of switching to a second tool. This leads to an ``adaptive" process whereby users will prefer leveraging existing tools in the analytics environment (no matter how well suited to the task) over switching to another application. Thus a well-designed interactive visual analysis environment will provide tools that enable users to perform such cleaning tasks as interactively as possible.


\section{Background} \label{sec:background}
\subsection{The ICU Date Format Language}

The choice of a date format syntax for generation is somewhat arbitrary, but for the purposes of exposition we will be using the formatting language defined by the ICU open-source project~\cite{ICU}. We chose this format because we were already using ICU in our system, we had access to the source code, and it provides localized date part data for a large number of languages (\eg month names). ICU's format syntax is typical of most syntax languages and provides a large number of date part codes, and a summary of it is reproduced in Table 2. The complete syntax is documented at the ICU web site.  Other syntaxes can either be generated directly from the algorithms, or by translating the ICU syntax into into the desired syntax. (In fact, this latter approach is how our visualization system internally generates formats for RDBMSes that have non-ICU syntax.)

\begin{table}[ht]
\centering
\bgroup
\def\arraystretch{1.5}
\begin{tabular}{|p{0.15\linewidth}| p{0.5\linewidth}|}
\hline
\textbf{ICU Code} & \textbf{Interpretation}\\ \hline

\scriptsize{yy} & \scriptsize{year (96)}\\ \hline
\scriptsize{yyyy} & \scriptsize{year (1996)}\\ \hline
    
\scriptsize{QQ} & \scriptsize{quarter (02)}\\ \hline
\scriptsize{QQQ} & \scriptsize{quarter (Q2)}\\ \hline
\scriptsize{QQQQ} & \scriptsize{quarter (2nd quarter)}\\ \hline
    
\scriptsize{MM} & \scriptsize{month in year (09)}\\ \hline
\scriptsize{MMM} & \scriptsize{month in year (Sept)}\\ \hline
\scriptsize{MMMM} & \scriptsize{month in year (September)}\\ \hline
    
\scriptsize{dd} & \scriptsize{day in month (02)}\\ \hline
    
\scriptsize{EEE} & \scriptsize{day of week (Tues)}\\ \hline
\scriptsize{EEEE} & \scriptsize{day of week (Tuesday)}\\ \hline
    
\scriptsize{a} & \scriptsize{am/pm marker (pm)}\\ \hline
    
\scriptsize{hh} & \scriptsize{hour in am/pm 1:12 (07)}\\ \hline
\scriptsize{HH} & \scriptsize{hour in day 0:23 (00)}\\ \hline
\scriptsize{mm} & \scriptsize{minute in hour (04)}\\ \hline
\scriptsize{ss} & \scriptsize{second in minute (05)}\\ \hline
    
\scriptsize{S} & \scriptsize{millisecond (2)}\\ \hline
\scriptsize{SS} & \scriptsize{millisecond (23)}\\ \hline
\scriptsize{SSS} & \scriptsize{millisecond (235)}\\ \hline

\scriptsize{zzz} & \scriptsize{Time Zone: specific non-location (PDT)}\\ \hline

\scriptsize{Z} & \scriptsize{Time Zone: RFC 822 (-0800)}\\ \hline
\scriptsize{ZZZZ} & \scriptsize{Time Zone: localized GMT (GMT-08:00)}\\ \hline
\scriptsize{ZZZZZ} & \scriptsize{Time Zone: ISO8601 (-08:00)}\\ \hline
    
\scriptsize{'} & \scriptsize{escape for text (nothing)}\\ \hline
\scriptsize{''} & \scriptsize{two single quotes produce one (')}\\ \hline

\end{tabular}
\egroup
\label{tab:icuformats} \\
\caption{ICU Format Codes.}
\end{table}

While fairly extensive, the ICU syntax has a few limitations that we ran into when working with real-world data:

\begin{itemize}
\setlength\itemsep{0em}
\item No support for 4 letter month abbreviations (\eg \texttt{Sept.})
\item No support for ordinal days (\eg \texttt{July 4th})
\item No support for quarter postfix notation (\eg \texttt{2Q})
\item No support for variant meridian markers (\eg \texttt{a.m.})
\end{itemize}

These limitations did not affect the results significantly, and in the future we hope to submit ICU extensions to handle some of these formats. Moreover, all the languages we examined had their own limitations, so ICU is not unusual in this respect.

One other quirk of the ICU syntax may be a contributing factor to the observed user difficulties writing correct ICU date formats in our visualization system. The use of lexicographical case in the meta-symbols of the format language can be confusing (\eg \texttt{y} is used for years, but \texttt{M} is used for months while \texttt{m} is used for minutes. A better design would have made case more consistent within dates and times. Another advantage of automated syntax generation is that it hides such problems from most users, significantly improving the usability of the function.

\subsection{The \dateparse\ Function}

The date format generated by the algorithms is presumed to be a string argument to a scalar function called \dateparse, which converts a string value to a date value using the format string. Several examples of this kind of function can be found in RDBMSes such as MySQL's \functionname{STR_TO_DATE}, Oracle's \functionname{TO_TIMESTAMP} and Postgres' \functionname{TO_TIMESTAMP}. There are also programming language library implementations such as Python's \functionname{strptime} and ICU's own \functionname{DateFormat::parse}.


\section{Minimal Descriptive Length} \label{sec:mdl}
The first algorithm is a Minimum Descriptive Length~\cite{Rissanen:1978} approach derived from the domain system presented in the Potter's Wheel system of Raman \textit{et al.}~\cite{Raman:2001}. We describe a number of extensions to their structure extraction system to support more complex redundancy, non-English locales, improved performance and date-specific pruning.

\subsection{Domains}
Potter's Wheel presents an algorithm for deriving a common structure for a set of strings by breaking each string down into a sequence of domains. These domains are described by an interface that includes:
\begin{itemize}
\item A required inclusion function to test for membership in the domain (\texttt{match})
\item An optional function to compute the number of values in the domain with a given length (\texttt{cardinality})
\item An optional function to update statistics for the domain based on a given value (\texttt{updateStatistics})
\item An optional function to prevent consideration of a domain that is redundant (\texttt{isRedundantAfter}).
\end{itemize}

In our approach, we implement all of these functions, but with significant changes to the last one, which we will describe below.

With this interface, we can now define a set of domains for each date part that we wish to be able to parse. These are mostly straightforward enumerations and numeric ranges, each tagged with the ICU format code. Since the ICU parser is flexible about parsing single or double digit formats, we use double-digit formats, but accept one or two digits. One important exception to this rule is for years, which are fixed width fields (2 or 4).

We also included some enumerated domains for handling constant strings and some simple regular expression domains for delimiters such as spaces, punctuation or alphabetic characters. We found that the inclusion of arbitrary numeric domains caused the run time to grow exponentially as the number of possible matches could not be pruned intelligently. This restriction extends to domains that can contain arbitrary digit sequences (such as any). Because of this restriction, the algorithm cannot extract non-date numeric fields.

\subsection{Redundancy Extensions}
A difficulty in using this kind of structure extraction is that the algorithm for enumerating structures is exponential in the number of domains. This is especially true in the date format problem because there are identical domains (\eg months and meridian hours), nearly identical domains (\eg days and hours) and there are often no field delimiters (\eg \texttt{2012Mar06134427}). To handle this, we have extended the existing pruning API with two other sets of domain identifiers:
\begin{itemize}
\item A set of \textit{prunable} identifiers, which are not allowed to precede the domain. For example, once we have a month field, no other month fields should be generated. Each month domain therefore lists all the month domains in its prunable set.
\item A set of \textit{context} identifiers, one of which must have been previously generated before the domain is considered. For example, a meridian domain can only be generated once an hour field has been found, but there may be other intervening fields.
\end{itemize}

\subsection{Performance}
Structure enumeration is computationally expensive, so we added a number of enhancements to the original domain extraction algorithm to keep the run time low enough for interactivity.

\subsubsection{Domain Characteristics}
Date domains typically have small widths, so we found it advantageous to provide the shortest and longest match sizes for use in structure enumeration and matching.

Date domains are also often uniform in that adding more characters to a mismatch will not help. For example, a 2-digit day domain that does not match a 1-letter substring will not be able to generate a match by adding more characters.

\subsubsection{Parallel Evaluation}
We have also identified two opportunities for parallel computation during structure enumeration.

The first computationally expensive operation is the enumeration of structures for a given sample. To parallelize this step, each thread is given a subset of the samples and independently produces a set of candidate domains. When all threads have completed, the duplicates are removed to produce a single list of candidate structures.

The second computationally expensive operation is the evaluation of each generated structure over the entire sample set. This includes computation of the MDL, recording of domain statistics and parameterizing the structure. These tasks are simple to parallelize, because there are no overlaps between the data for each structure.

\subsection{Unparameterization}
Domain parameterization is an important part of generating compact representations via MDL, but it creates problems for date recognition. For example, if a set of dates contains a constant month string (\eg all values are in September) it is important to keep track of the month name domain. Consequently, when we parameterize a constant generic \texttt{<Word>} domain, we tag it with the date part domain that it matches (if any). We then need to apply an additional pruning step to remove any structures that also found an equivalent domain (\eg two-digit month). These rules are equivalent to the context-based redundancy rules above, but have to be applied again after parameterization of generics.

\subsection{Global Pruning}
The pruning rules used for the structure extraction reduce the search space dramatically, but they are also contextual and can only look backwards. The domains also contain a fair amount of ambiguity that requires the application of domain knowledge. We therefore found it necessary to add some post-generation global pruning rules:
\begin{itemize}
\item The set of date parts cannot contain place value gaps (\eg structures that have year and day without month are removed.)
\item Similarly, the set of time parts cannot contain place value gaps and must also be in place value order (times are never written in orders such as \texttt{mhs}.)
\item The existence of time parts cannot make dates incomplete (\eg patterns like year-day-hour are removed.)
\item Two digit years require special handling. In particular, they cannot appear adjacent to a two-digit field if the structure contains punctuation. (This can come up in some small early 21st century year domains where a two-digit year can masquerade as almost any numeric field.)
\end{itemize}

These global pruning rules are simple and intuitive, but are essential to further reducing the search space.

\subsection{Locale Sensitivity}
Providing an acceptable international user experience requires correctly handling the locale of the column text. Accordingly, at the start of the structure extraction we use the column locale to create a set of domains containing locale-sensitive strings such as month names. We also use the locale to map these strings to upper- and lower-case in addition to the ICU mixed case strings. This enables us to accurately compute MDL statistics without having to map the candidate strings at runtime (which would be slow).

Knowing the locale of a string is not always helpful. In our data set, we found numerous cases where the locale was specified (\eg Sweden) but the data was actually in English. Accordingly, we test both locales and rank the combined results. 

We have built this system for the Gregorian calendar only as our visualization system does not support non-Gregorian calendars and we have little evidence of other calendars (\eg Hebrew, Islamic Civil) being used for analytics. ICU supports non-Gregorian calendars and we expect the algorithm could be extended to them as well if needed.

\subsection{Ranking}
MDL structure analysis naturally produces a ranked list of format candidates, but we have found that a number of other properties of the formats should be preferred over simple compactness:
\begin{itemize}
\item Since we have a set of samples, we can apply the candidate format to the strings to see how well it performs. Formats with fewer parse errors are preferred.
\item Date parts can be considered a place-value system, so we prefer ``more significant'' components (\eg month-day-year over hour-minute-second).
\item If two formats from different locales give the same results, prefer the original column locale. The sample set may have missed an example where this could be important.
\item If the format has an ambiguous date order (\eg all days are less than 12), then prefer the default date order of the locale. Again, the sample set may have missed a counterexample, so this is the best option.
\item Once these semantic preferences have been considered, we then prefer the more compact (MDL) representation.
The output of the algorithm is now an ordered list of formats and associated locales. These can then be used to drive a user interface that allows the user to choose between the possibilities or the top-ranking format can simply be used automatically.
\end{itemize}

\section{Natural Language Processing} \label{sec:nlp}
(motivate why an NLP approach was considered. Also add technical detail on programming environment and parser used for grammar)

\subsection{Context-Free Grammar}
ICU date-time formats are well defined both structurally and semantically, and can be defined by a context-free grammar (CFG). A CFG is commonly defined as a set of productions or rules of the form A ? ? where A is a variable, ? is a sequence of variables and terminal symbols (the tokens that make up the alphabet of the language) plus null (? ), and the production symbol (?) indicates that the variable A can be expanded into ? . A CFG can be formally specified with four components: V, T, P, and S, where V is the set of variables, T the set of terminal symbols, P the set of productions, and S the set of available start symbols (a non-empty subset of V) [cite]. 

While there are several functionally equivalent notations for representing CFG, we use the \textit{Backus-Naur Form} (BNF) for defining the grammar rules for \dateparse formats. In particular, we use the \textit{Extended Backus-Naur Form} (EBNF) as the notation is more compact and readable for frequently used constructions~\cite{Grune:1990}. 

We define a grammar for identifying date-time strings based on other EBNF based date-time formats [cite] as a reference. A partial definition of the grammar is found below (see supplementary material for complete definition):

\begin{grammar}
<TimeGrammar> ::= <Hours> ':' <Minutes> ':' <Seconds> ;\\

<DateGrammar> ::= <BigEndianDate> 
				\alt <MiddleEndianDate> 
				\alt <LittleEndianDate>;

<DateTimeGrammar>  ::= <DateGrammar> 
					\alt <TimeGrammar>;
					

<BigEndianDate> ::= <Year> <Month>  <Day> ;

<MiddleEndianDate> ::= <Month> <Day> <Year>;

<LittleEndianDate> ::= <Day> <Month> <Year>;

<Year> ::= <TwoYear> | <FourYear>;

<Month>  ::= <MonthFullForm> | <MonthAbbrForm> | <MonthLetterForm> | <MonthNumber>;

<Day>     ::= d d (between 01 and 28-31, depending on month/year);

<Hour> ::= <TwelveHour> | <TwentyFourHour>

<TwelveHour> ::= d d (between 00-12);

<TwentyFourHour> ::= d d (between 00 and 23);

<d> ::= `0' | `1' | `2' | `3' | `4' | `5' | `6' | `7' | `8' | `9'

\end{grammar}



\subsection{Translation to ICU Format}

\subsection{Production Rule Constraints and Variants}

The EBNF date-time grammar includes a large number of syntactically correct but semantically invalid
date-time expressions. While we have added range restrictions to symbols such as \texttt{Hour} (1--12 for 12-hour format and 1--24 for 24-hour format), \texttt{Days} (1--7), \texttt{Month} (1--12), there are special cases that need to be accounted for. For example, there is no  `November 31, 2015', `February 29, 2013', or `Sunday, May  5, 1965'. November only as 30 days in any year; 2013 was not a leap year; and May 5, 1965 was a Wednesday.

While custom production rules can be added to the existing grammar to exclude such expressions, this approach is not optimal as it leads to a rather large grammar that needs to account for every single semantically valid date-time sequence of terminal symbols. Rather, we modify the existing grammar with the following additional constraints to the \texttt{Day} terminal symbol for excluding such expressions:

\begin{itemize}
\item Restriction on the distribution of $30$ and $31$:
Months usually alternate between lengths of 30 and 31 days. We can use x mod 2 to get an alternating pattern of 1 and 0, then just add our constant base number of days:

\begin{equation}
\texttt{Day} = 30 + x \Mod 2
\end{equation}

where $x \in [1..12]$ for each of the $12$ months.

Except February, Equation $1$ addresses January and March through July. After July, the pattern should skip one, and the rest of the months should follow the alternating pattern inversely.

To obtain an inverse pattern of alternating 0 and 1, we add 1 to the dividend:

\begin{equation}
\texttt{Day} = 30 + (x + 1) \Mod 2
\end{equation}

In Equation $2$ the number of days for August through December is correct ($x \in [8..12]$), but not for the remaining months. We hence introduce a bit-masking function so that the equation is equal to 1 over the desired domain and 0 otherwise. Multiplying a term by this expression will result in the term being cancelled out outside its domain. To mask the latter piece of our function, we need an expression equal to $1$ where $8 \le x\le 12$. Floor division by 8 works well, since $x < 16$.

Now if we substitute this expression in the x + 1 dividend of Equation $2$, we can invert the pattern using our mask:

\begin{equation}
\texttt{Day} = 30 + (x + \floor{\frac{x}{8}}) \Mod 2
\end{equation}


\item LeapYear Restriction for February:
While the above restriction applies to all months barring February, we also apply a constraint to the number of days for February, based on whether the year is leap year or not. For this, we define a new symbol in the grammar called \texttt{LeapYear}. If an expression containing the month `February' or any such variant (\textit{e.g.} `Feb', `2') with the day `29' and a year, would need to resolve the \texttt{Year} symbol to be a \texttt{LeapYear}, defined as:

\begin{equation}
\texttt{Year} \Mod 4 == 0
\end{equation}

\end{itemize}

Equations $3$ and $4$ are now added as constraints to the \texttt{Days} symbol in the grammar.

\subsection{Probabilistic Context-Free Grammar}
Pattern-recognition problems such as parsing date and time formats initiate from observations generated by some structured stochastic process. In other words, even if the initial higher-level production rule of the grammar is known (i.e. date, time or date-time), there could be several directions that the parser resolves to. For example, in a date string \texttt{5/6/2015}, the pattern could either be \texttt{M/d/yyyy} or \texttt{d/M/yyyy}. 

In the context of CFGs, probabilities have been used to define a probability distribution over a set of parse trees defined by the CFG, and are a useful method for modeling such ambiguity~\cite{Collins:2003,Manning:1999}. The resulting formalism called Probabilistic Context-Free Grammar (PCFG), extends the CFG by assigning probabilities to the production rules of the grammar. During the process of parsing the date-time pattern, the probabilities are used as a filtering mechanism to rank the pattern(s) that a given string resolves to in the grammar. The parser then chooses the parse tree with the maximum probability.

(add the rules for how probabilistic weights are assigned to the CFG)

\subsection{Columnar Context}

Once we have created a PCFG model of a process, we can apply existing PCFG parsing algorithms to identify a variety of date-time formats. However, the parser's success is often limited in the types of the dominant patterns that it can identify. In addition, the standard parsing techniques generally require specification of a complete observation sequence. In many contexts, we may have only a partial sequence available (\textit{e.g.} an incomplete entry). Finally, we may be interested in computing the probabilities of date-time patterns that the grammar may not explicitly define. To extend the forms of evidence, inferences, and pattern distributions supported, we need a flexible and expressive representation for the distribution of structures generated by the grammar. We adopt Bayesian networks for this purpose, and define an algorithm to generate a probabilistic distribution of possible parse trees corresponding to a set of date-time patterns as opposed to individual ones. 


\subsection{Semantic Extensions}

%�	Explain why we use a probabilistic version of the grammar
%�	Explain how the production rules are created based on the ICU format
%�	Extensions to the parser to account for columnar context for maximizing the probabilistic occurrence of the dominant pattern
%�	Other semantic extensions � looking at the file name, adding rules for leap year, using external dictionaries for non-English terms


\section{Evaluation} \label{sec:experiments}
\subsection{Data}

\subsubsection{Training Data}
All files published to Tableau Public through February 2014 were scanned. We collected the contents of columns with names containing any of the following strings: Date, month, created, dt (abbreviation for date), mes (month in Spanish), datum (date in German), fecha (date in Spanish), data (date in Portuguese), ? (day in Chinese and Japanese). Roughly 95\% of the data in Tableau Public is in English, but we attempted to include non-English data that was available. Fields of any data type other than date or datetime were analyzed, including strings, integers, and floats. One file was created for each scanned column, containing the unique non-null values in the column.

The resulting set of files were de-duplicated. This handles any duplicate files on Tableau Public including sample data. Files over 1MB were manually reviewed and those that did not contain dates removed. There were 30,968 files in the resulting set.
Most database and spreadsheet systems already detect a limited set of date formats. For instance, typing the string \texttt{"12/31/1999"} into Microsoft Excel, is automatically interpreted as the date \texttt{1999-12-31}. The Microsoft Jet library that Tableau used to read these text files detects a few date formats as well. Any column already converted by Excel or Jet was not included in this study.

\textbf{Verification Data}\\
Verification used another set of data from Tableau Public, collected through April 2015. This set was collected similarly, but also added columns named time. We limited this set to new Tableau Excel and text connectors that were released in May 2014. It is likely that some of these files were older files republished to Tableau Public. There were 31,546 files in the resulting set.

The new Tableau text connector is able to automatically interpret a somewhat different limited set of date formats. Any column already converted by Excel or the new Tableau text connector was not included in this study.

\subsection{Evaluation}
Once the training data was analyzed, it was grouped by date format. A sample of each produced date format was manually labeled. This allowed us to quickly skip over very common formats like \texttt{MM/dd/yyyy} and focus our efforts on much less common formats. The samples were judged as to whether the produced format was reasonable and were tagged with correct formats if the produced format was unreasonable.

Further manual tagging focused on the files most likely to represent dates. Many of the files in the collection are not actually dates, such as fields named "updated by" (which contain "date"). A random sample of 850 columns named exactly "date", "time" or "month" (case-insensitive) were manually judged.

\subsubsection{	Minimum Descriptive Length}
Testing of the MDL algorithm was performed on a 24-core Dell T7610 running Windows 7 with the data stored on a 250GB SSD.
 
%Figure M1. MDL Parsing Statistics
To test the MDL algorithm, we ran it over the set of samples from each validation file to generate a ranked list of formats for the file. Each format was then applied to the entire file�s data set, recording both the number of errors and the elapsed times. In cases where we generated multiple formats and the main format produced errors, we applied the second format to the unparsed strings. The summary statistics from this processing are presented in Figure M1.

The analysis speed is the average time needed per sample for structure extraction. At 2.5ms, this is well below most human perceptual thresholds for a set of 32 samples, so any latency in command execution would be restricted to the ability of the underlying database to provide the samples for analysis in a timely manner. (As a column store, the TDE can often supply such domains without a full table scan, further improving responsiveness.) 

The validation speed is the average time needed to parse a value, and provides an estimate of how fast an ICU-based implementation (such as the TDE) can process string values into scalars and works out to 620K values per core per second.
 
%Figure M2. MDL Error Rate
The error rate reflects the fact that only about 40\% of the files have an associated format that parses the non-null values without error. To examine the error rate in greater detail, we turn now to Figure M2.

On the left hand side, we can see that the algorithm found 744 distinct formats that parsed 13,424 files with no error. This is a remarkable number of valid formats and underscores the need for this kind of algorithm. Raising the error rate threshold to 5\% results in about 2500 formats found in 15,000 files, or nearly half of the files. 

What do these formats look like? Figure M3 shows a histogram of the 25 most common formats containing a year format code at the 5\% threshold, color-coded by error rate. (A sample value is provided to the right of each bar for illustrative purposes.) The formats have also been filtered to files with at least 5 samples. Most of the samples are clearly dates with a wide range of formats (the format where the time zone is between the time and the year is surprisingly common.)
 
%Figure M3. MDL Output
Some of the dates are clearly just numbers, but our approach is to assume that when the user tells us that the column contains dates we should find the best fit. The samples include dates from a wide range of historical sources (e.g. Roman pottery dates) so we have elected to defer the date identification task to the user.

\subsubsection{Natural Language Processing}

\subsubsection{	Cross-Checking}


\section{Future Work} \label{sec:future}
During testing, we found a number of columns that we were not able to process with these techniques. Many columns contained multiple date formats. We would like to recognize this situation and generate predicate-based calculations (\eg if the string matches format 1 then apply format 1 else ...) to increase the accuracy of the results and thereby make the experience even more seamless for the user. Other columns contain date ranges, which we would like to handle by generating multiple calculations, possibly by combining regular expressions with date parsing.

In this paper, we have only considered the parsing of strings, but dates are often formatted as integers (\eg \texttt{201507016}). It is significantly faster to decompose integers into date parts using arithmetic operations (\eg \texttt{mod} and \texttt{div}) than by using slower, locale-sensitive string parsing functions. Moreover, the number of possible formats is low enough that we may be able to enumerate them. Timestamp preparation from numeric representations is another form of preparation which we would like to automate.

In the course of our research, we have also identified a number of date part variants (\eg ordinal dates, four-letter month abbreviations, alternate meridian markers and postfix quarter syntax) that we would like to commend to the ICU project, along with possible implementations.


\section{Conclusion} \label{sec:conclusion}
In this paper, we have described two effective algorithms for extracting date format strings from a small set of samples, one using a minimum descriptive length approach and one using natural language techniques. Both algorithms are accurate enough to be used automatically without user involvement. The MDL algorithm is also fast enough to deploy in an interactive environment.

While validating the algorithms on a large corpus, we also found that the number of distinct formats in the wild is surprisingly high, and demonstrates the wisdom of including general-purpose date parsing functions in data visualization tools, data cleaning tools and RDBMSes. In particular, it is interesting to note that the most prominent open source RDBMSes (\eg MySQL and Postgres) both have a built-in version of \texttt{DATEPARSE}, possibly reflecting that this is a common need that gets implemented when users are empowered to extend the function library of an RDBMS.


%ACKNOWLEDGMENTS are optional
%\section{Acknowledgments}
%We would like to thank to Douglas Adams for encouraging this work; BBC Wales for providing the resources to productize it; and Doctor Who Live, whose harrowing live demonstration of episode production in front of several thousand people inspired this project.


%
% The following two commands are all you need in the
% initial runs of your .tex file to
% produce the bibliography for the citations in your paper.
\bibliographystyle{abbrv}
\bibliography{dateparse}  % sigproc.bib is the name of the Bibliography in this case
% You must have a proper ``.bib'' file
%  and remember to run:
% latex bibtex latex latex
% to resolve all references
%
% ACM needs 'a single self-contained file'!
%

%\balancecolumns
% That's all folks!
\end{document}
