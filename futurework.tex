While we are satisfied with the results so far, we found a number of column that we were not able to process with these techniques. Many columns contained multiple date formats, and we would like to recognize this situation and generate predicate-based calculations to increase the accuracy of the results and thereby make the experience even more seamless for the user. Other columns contain date ranges, which we would like to handle by generating multiple calculations, possibly by combining regular expressions with date parsing.

In this paper, we have only considered the parsing of strings, but dates are often formatted as integers (\textit{e.g.} \texttt{201507016}). It is significantly faster to decompose integers into date parts using arithmetic operations (\textit{e.g.} \texttt{mod} and \texttt{div}) than by using slower, locale-sensitive string parsing functions. Moreover, the number of possible formats is low enough that we may be able to enumerate them. Timestamp preparation from numeric representations is another form of preparation which we would like to automate.

In the course of our research, we have also identified a number of date part variants (\textit{e.g.} ordinal dates, four-letter month abbreviations, alternate meridian markers and postfix quarter syntax) that we would like to commend to the ICU project and maybe provide implementations of.
