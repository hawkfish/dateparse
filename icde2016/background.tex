\subsection{The ICU Date Format Language}
The choice of a date format syntax for generation is somewhat arbitrary, but for the purposes of exposition we will be using the ICU Date Format Language. Other syntaxes can be generated directly from the algorithms, or by translating the ICU syntax into 

The date format syntax we selected for generation is the one provided by the ICU open-source project. We chose it because we were already using ICU in the code base, we had access to the source code, and it provides localized date part data for a large number of languages. 

ICU's format syntax is typical of most syntax languages and provides a large number of codes, and a summary of it is reproduced in Table 2.

\begin{table}[ht]
\centering
\begin{tabular}{|p{0.2\linewidth}| p{0.6\linewidth}|}
\hline
\centering
\textbf{ICU Code} & \textbf{Interpretation}\\ \hline

\scriptsize{yy} & \scriptsize{year (96)}\\ \hline
\scriptsize{yyyy} & \scriptsize{year (1996)}\\ \hline
    
\scriptsize{QQ} & \scriptsize{quarter (02)}\\ \hline
\scriptsize{QQQ} & \scriptsize{quarter (Q2)}\\ \hline
\scriptsize{QQQQ} & \scriptsize{quarter (2nd quarter)}\\ \hline
    
\scriptsize{MM} & \scriptsize{month in year (09)}\\ \hline
\scriptsize{MMM} & \scriptsize{month in year (Sept)}\\ \hline
\scriptsize{MMMM} & \scriptsize{month in year (September)}\\ \hline
    
\scriptsize{d} & \scriptsize{day in month (2)}\\ \hline
\scriptsize{dd} & \scriptsize{day in month (02)}\\ \hline
    
\scriptsize{EEE} & \scriptsize{day of week (Tues)}\\ \hline
\scriptsize{EEEE} & \scriptsize{day of week (Tuesday)}\\ \hline
    
\scriptsize{a} & \scriptsize{am/pm marker (pm)}\\ \hline
    
\scriptsize{h} & \scriptsize{hour in am/pm 1:12 (7)}\\ \hline
\scriptsize{hh} & \scriptsize{hour in am/pm 1:12 (07)}\\ \hline
    
\scriptsize{H} & \scriptsize{hour in day 0:23 (0)}\\ \hline
\scriptsize{HH} & \scriptsize{hour in day 0:23 (00)}\\ \hline
    
\scriptsize{m} & \scriptsize{minute in hour (4)}\\ \hline
\scriptsize{mm} & \scriptsize{minute in hour (04)}\\ \hline
    
\scriptsize{s} & \scriptsize{second in minute (5)}\\ \hline
\scriptsize{ss} & \scriptsize{second in minute (05)}\\ \hline
    
\scriptsize{S} & \scriptsize{millisecond (2)}\\ \hline
\scriptsize{SS} & \scriptsize{millisecond (23)}\\ \hline
\scriptsize{SSS} & \scriptsize{millisecond (235)}\\ \hline
    
\scriptsize{Z} & \scriptsize{Time Zone: RFC 822 (-0800)}\\ \hline
\scriptsize{ZZ} & \scriptsize{Time Zone: RFC 822 (-0800)}\\ \hline
\scriptsize{ZZZ} & \scriptsize{Time Zone: RFC 822 (-0800)}\\ \hline
\scriptsize{ZZZZ} & \scriptsize{Time Zone: localized GMT (GMT-08:00)}\\ \hline
\scriptsize{ZZZZZ} & \scriptsize{Time Zone: ISO8601 (-08:00)}\\ \hline
    
\scriptsize{'} & \scriptsize{escape for text (nothing)}\\ \hline
\scriptsize{''} & \scriptsize{two single quotes produce one (')}\\ \hline

\end{tabular}
\label{tab:icuformats} \\
\caption{ICU Format Codes.}
\end{table}

The complete syntax is documented at the ICU web site~\cite{ICU}. 

While fairly extensive, the ICU syntax has a few limitations that we ran into when working with real-world data:
\begin{itemize}
\item No support for 4 letter month abbreviations (\eg \texttt{Sept.})
\item No support for ordinal days (\eg \texttt{July 4th})
\item No support for quarter postfix notation (\eg \texttt{2Q})
\item No support for variant meridian markers (\eg \texttt{a.m.})
\end{itemize}

These limitations did not affect the results significantly, and in the future we hope to submit ICU extensions to handle some of these issues. Moreover, all the languages we examined had their own limitations, so ICU is not unusual in this respect.

One other quirk of the ICU syntax may be a contributing factor to the user confusion around writing correct ICU date formats in the existing visualisation system. The use of lexicographical case in the meta-symbols of the format language can be confusing (\eg \texttt{y} is used for years, but \texttt{M} is used for months while \texttt{m} is used for minutes. A better design would have made case more consistent within dates and times. Another advantage of automated syntax generation is that it hides such problems from most users, significantly improving the usability of the function.
